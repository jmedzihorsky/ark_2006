\documentclass[11pt, a4paper]{article}

\usepackage{a4wide}

\usepackage{datetime}
\usepackage{hyperref}
\usepackage{booktabs}
\usepackage{float}

\usepackage{charter}
\usepackage[T1]{fontenc}


\title{Atlas of Romani Communities in Slovakia 2006 Update Dataset}
\author{
	Juraj Medzihorsky \\
	\href{mailto:medzihorsky_juraj@ceu-budapest.edu}{\texttt{medzihorsky\_juraj@ceu-budapest.edu}}
}
\date{\ddmmyyyydate\today~at~\currenttime}

\begin{document}

\maketitle

\noindent
\textbf{Cite the data \& codebook as:}

\begin{itemize}
	\item
Medzihorsky, Juraj. (2013) \textsl{Atlas of Romani Communities in Slovakia 2006 Update Dataset}. version: \ddmmyyyydate\today~at~\currenttime
\end{itemize}

\noindent
\textbf{The original data:}

\begin{itemize}
	\item
Social Policy Analysis Center (SPACE). (2006). \textsl{Aktualiz\'acia Atlasu r\'omskych komun\'it na Slovensku}
\end{itemize}



\noindent
\textbf{The first edition of the Atlas:}

\begin{itemize}
	\item
Radi\v{c}ov\'a, Iveta. (2004). \textsl{Atlas r\'omskych komun\'it na Slovensku 2004}. [Atlas of Roma communities in Slovakia]. Bratislava: SPACE, IVO, KcpRO.
\end{itemize}


\noindent


\tableofcontents


\section{Sources}

\begin{itemize}

	\item	Settlement data were scraped from \href{http://www.nspace.sk/romska_komunita.html}{http://www.nspace.sk/romska\_komunita.html}.

	\item 	Manual for the original data is available at \href{http://www.nspace.sk/files/atlas_osf_2006.pdf}{http://www.nspace.sk/files/atlas\_osf\_2006.pdf}.

	\item	Official census territorial codes were scraped from \href{http://www.sodbtn.sk/obce/abeceda.php}{http://www.sodbtn.sk/obce/abeceda.php}

\end{itemize}


\section{Changes}

\begin{enumerate}
	\item 	Special characters removed, everything converted to ASCII encoding.
	\item 	County name for municipality `Krasnohorske Podhradie' corrected from `Kosice - Okolie'
			to `Roznava'
	\item 	Added census territorial codes.	
	\item 	Value labels Translated into English.
\end{enumerate}



\section{Waves}

\begin{itemize}
	\item	First wave:  2004.
	\item	Second wave:  2005.
\end{itemize}


\section{Territorial Units}

The 2006 update of the Atlas was focused on segregated communities.  
Out of the 281 segregated communities covered by the first edition
the update focused on 207 communities [the original manual states 
200, but the dataset covers 207 municipalities], excluding communities 
in Bratislava and Trencin regions, and city ghettos. 


\begin{table}[H]
	\label{tab:territ}
	\caption{Territorial units: Slovak \& English terms and counts.}
	\begin{center}
		\begin{tabular}{l l r}
			\toprule
			Slovak & English & $n$ \\
			\midrule
			kraj	&	region 							&	6	\\
			okres	&	county 							&	40	\\
			obec 	&	municipality					&	207	\\
			\bottomrule
		\end{tabular}
	\end{center}
\end{table}

The dataset contains settlements from the following
six out of eight regions:

\begin{table}[H]
	\label{tab:regions}
	\caption{Included municipalities.}
	\begin{center}
		\begin{tabular}{l r}
			\toprule
				Region & No. of Municipalities \\ 
			\midrule
		Banskobystricky kraj&   82	\\
        		Kosicky kraj&  148  \\	
		     Nitriansky kraj&    4  \\
      		  Presovsky kraj&  160  \\
               Trnavsky kraj&   14  \\
               Zilinsky kraj&    8  \\
			\bottomrule
		\end{tabular}
	\end{center}
\end{table}




\section{Coding}

Variable names are modular, see Table 3 for details.

\begin{table}[H]
	\label{tab:modules}
	\caption{Basic variable name elements.}
	\begin{center}
		\begin{tabular}{ll}
			\toprule
			Element	& Description \\		
			\midrule
				\texttt{sodbtn\_}	&	census territorial code 		\\
				\texttt{name\_}		&	official territorial name 		\\
				\texttt{type\_}		&	municipality type				\\
				\texttt{loc\_}		&	location of Romani settlement	\\	
				\texttt{inf\_}		&	infrastructure variable			\\
				\texttt{rom\_}		&	Romani settlement variable		\\
				\texttt{maj\_}		&	majority (non-Romani) part of 	\\
									&	the municipality variable		\\
			\bottomrule
		\end{tabular}
	\end{center}
\end{table}


\end{document}
